% first example chapter
% @author Andre Alves
%
\chapter{Editing Components}\label{ch:editing-components}

While the list of components was already adjusted to match those that are part of CircuiTikz, the CircuiTikz project is constantly adding new components.
Consequently, it is important to understand how components were removed from draw.io so they can be readded in the future.

\section{Finding draw.io components within CircuiTikz manual}\label{sec:finding-draw.io-components-within-circuitikz-manual}
There are two types of components in CircuiTikz, node type and path type.
The information about all the components along with their type and code to draw them is available within Circuitikz manual.\\
To find available draw.io components in CircuiTikz manual, the following steps were performed:
\begin{enumerate}
    \item Open \emph{CircuiTikz} manual and the Draw.io source code.
    \item In draw.io look at the component's name and shape.
    \item Try to find the relevant section in CircuiTikz manual and try to match the name and shape of the component.
\end{enumerate}

\section{Removing components from the sidebar which are not a part of CircuiTikz}\label{sec:removing-components-from-the-sidebar-which-are-not-a-part-of-circuitikz}
To add/remove components from sidebar following steps were taken:
\begin{enumerate}
    \item Open \emph{src/main/webapp/js/diagramly/sidebar/Sidebar-Electrical.js} file.
    \item Look for the pallet in which the component belongs.
        All the components of a pallet are located within \emph{addPaletteFunctions} of that particular pallet.
    \item To remove the component from the sidebar, remove the \emph{createVertexTemplateEntry} line of the component within \emph{addPaletteFunctions}.
\end{enumerate}
