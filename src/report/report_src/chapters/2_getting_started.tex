% first example chapter
% @author Andre Alves
%
\chapter{Getting Started}\label{ch:getting-started}
Running the current webapp does not require installation, but it does require a functioning webserver.
Based on the advice from~\ref{subsec:intellij}, we recommend using IntelliJ's built-in web server.

\section{Prerequisites}\label{sec:prerequisites}
The following prerequisites are outside the scope of this section, but are necessary to follow our instructions on getting started.
\begin{itemize}
    \item A valid GitHub account.
    \item Git installed on your computer.
    \item Either GitHub Desktop installed or familiarity with using Git from the command line.
    \item An IDE that supports Apache Ant.
        We recommend \href{https://www.jetbrains.com/idea/download/}{IntelliJ IDEA Ultimate}.
\end{itemize}

\section{Steps}\label{sec:getting-started-steps}
While there are multiple ways to get started, this is our recommended approach.
\begin{enumerate}
    \item Fork the repository for the project (\href{https://github.com/andre-a-alves/drawio-circuitikz}{https://github.com/andre-a-alves/drawio-circuitikz}) so you are the project owner.
    \item Clone your new fork onto your local machine.
    \item Open the project in IntelliJ IDEA Ultimate.
    \item Within the project, right click \texttt{etc/build/build.xml} and click \texttt{Add as Ant Build File}.
\end{enumerate}

You will now see an Ant toolbar on the right hand side of IntelliJ.
By clicking that toolbar, you will see various build options.
By selecting \texttt{all} and clicking the green ``play'' arrow, you will be able to transpile your current source code into minimized javascript that can be run on a local server.

\section{Running the Webapp}\label{sec:running-the-webapp}
After following the above steps, to run the webapp, simple right click on \\\texttt{src/main/webapp/index.html} and select ``Run index.html.''
This will launch a local webserver that hosts the current site.

\section{Deploying the Webapp}\label{sec:deploying-the-webapp}
This project ended at webapp development.
Deploying the completed webapp is another topic remaining to be solved.