% first example chapter
% @author Andre Alves
%
\chapter{Parser Structure and Next Steps}\label{ch:parser-structure}


\section{Parser Operation}\label{sec:parser-ops}
The parser is at the heart of this team's additions to the original draw.io webapp and is responsible for processing an arbitrary circuit drawn by the user and generating the appropriate \LaTeX code for the CircuiTikz package to draw an identical circuit in \LaTeX.
The parser operates in two steps, with two major functions.
The first converts the XML structure which is used by draw.io into equivalent JavaScript objects in an object-oriented structure, and the second converts the JavaScript objects into \LaTeX code.\\

Draw.io uses an XML structure to represent the circuit drawn in its GUI, and the XML details each individual object and relevant metadata, such as the size and position of the object, whether the object is rotated, etc.
The XML structure is updated every time a change is made, such as when a resistor is shifted sideways or when a new capacitor is added.
The first parsing function was added to this event, so that the XML-to-JavaScript function is called whenever the XML is updated.
This function takes the XML and parses it to create JavaScript objects. \\

Different JavaScript classes are used, which correspond to the Path-style, Node-style, and simple line components in CircuiTikz.
A lookup table is used to add CircuiTikz-relevant data to the JavaScript objects, so that later when writing \LaTeX code, all the required information is already packaged into the objects.
All these JavaScript objects are saved to a map, with the draw.io ID as the key and the object itself as the value.\\

The second parsing function takes the map of JavaScript objects and generates \LaTeX code from it.
This function is called only when the user requests the circuit be exported to \LaTeX.
The export to \LaTeX button can be found in the File menu bar.
When called, the JavaScript-to-LaTeX function takes each JavaScript object from the map and writes the appropriate code to draw this with the CircuiTikz package in \LaTeX.
As of this report, the JavaScript-to-LaTeX function is functional for path-style and simple line components.


\section{Parser Functions}\label{sec:parser-funcs}
\begin{itemize}
    \item \texttt{parseXmlToCircuitMap()}: Automatically called by draw.io whenever the XML structure is changed (i.e.\ whenever the circuit in the GUI is edited).
        Parses and converts draw.io's XML into JavaScript objects which are saved in a map.
        These objects contain all the relevant data from XML and also data that is necessary when converting into CircuiTikz code, such as the name of this component in CircuiTikz.
        Unnecessary data from the XML are omitted.
    \item \texttt{createCkt()}: Called when the user manually requests the current circuit to be exported to \LaTeX code.
        This function takes the current JavaScript map representation of the circuit and returns the correct CircuiTikz code to draw the circuit in \LaTeX.
        The code block for processing CircuiTikz Node-style components is incomplete, but the function will run without errors when executed.
    \item \texttt{adjust()}: Adjusts the x and y coordinates in draw.io for use in CircuiTikz.
        For example, in draw.io, the relative origin of a component is the upper-left corner, but in CircuiTikz this is the lower-left corner.
        Also, draw.io uses pixels to represent the position and dimensions of components, while CircuiTikz uses fractional values, such as 1 or 1.5.
        This function is responsible for adjusting all the x and y coordinates as necessary to correctly position the component in CircuiTikz.
    \item \texttt{getCktOrigin()}: Finds a relative origin point for a draw.io circuit, which will be used as the origin point for drawing the circuit in CircuiTikz.
        The origin point is determined so that the point is to the lower-left of the entire circuit, i.e. the origin point is to the left and below all components of the circuit.
    \item \texttt{getRotatedVertices()}: Calculates the specific x and y coordinates of rotated draw.io components.
        Rotated and unrotated draw.io components only specify their x and y coordinates, height, width, and degree of rotation, so this function is required to calculate the exact coordinates for the components when rotated.
        If there is no rotation, the calculated coordinates match the original coordinates.
    \item \texttt{getEndPoints()}: Calculates the end points of lines (transmission paths) in draw.io, which are the exact coordinates where a line meets a component
\end{itemize}


\section{Lookup Table}\label{sec:parser-lookup-table}
The parser relies on a lookup table to match draw.io components to CircuiTikz components.
When the parser is converting draw.io's XML structure to JavaScript objects, it must find out if this component is a Path-style component, Node-style component, or a simple short circuit (transmission path) in CircuiTikz and determine the exact code to use in CircuiTikz to draw this component.
The lookup table provides the information for this and is provided in three separate CSV files, \emph{circuitikzshape.csv}, \emph{drawioshape.csv}, and \emph{objtype.csv}, which can be found in the same directory as the parser.
These CSV files are read after the window finishes loading and saved in a JavaScript map object at runtime.
The data in the CSV files come from the \emph{CircuitikzShape}, \emph{DrawIOShape}, and \emph{ObjType} columns in the Excel file \emph{drawio\_circuitikz\_v19082022.xlsx}, which can be found in the same directory.
When updating the CSV files, simply order the Excel file using the \emph{No} column (first column on the left) and copy-paste each column into the matching CSV file.
The original intention was to read directly from the Excel file, but CSV files were used instead because of difficulties in reading only specific columns from the Excel file.


\section{Next Steps}\label{sec:pick-up}
The JavaScript-to-LaTeX function is not correctly functioning when processing CiruiTikz Node-style components.
This is because Node-style components such as logic gates or transistors have named points at specific locations.
In contrast, Path-style components such as resistors or inductors have nameless points and can be drawn within the same code as short circuits (transmission paths).
The current code does not properly connect Node-style components to short circuits when converted into CircuiTikz code, with gaps between the component ends and the short circuits.
A possible solution which could not be implemented in time was to add the specific end-points in CircuiTikz and end-point locations in draw.io to the lookup table and have the function identify the connections while processing Node-style components.
An edited Excel file matching named pins in CircuiTikz and the locations of these pins in draw.io components can be found with the source code for the parser.
This file is the same as the Excel file used for the lookup table, except for one additional column.
The name of the file is \emph{drawio\_circuitikz\_v29082022.xlsx}.
Please note that the current lookup table uses the data from the Excel file \emph{drawio\_circuitikz\_v19082022.xlsx}.
