% first example chapter
% @author Andre Alves
%
\chapter{Introduction}\label{ch:introduction}
As a CJ1 project, this team chose to build a webapp that would allow a user to draw a circuit graphically and use that circuit to obtain \LaTeX code to draw a circuit using the CiruiTikZ package.
While the project was subject to a number of delays, a functional MVP was developed, and its source code is publicly available on GitHub.
Instead of building the webapp from scratch, the team forked the \emph{draw.io} software by JGraph Ltd~\cite{drawioCode}, which was renamed \emph{diagrams.net} while working on this project.
While starting from that source code was essentially necessary due to the scope of the project, this came with many of its own challenges.

\section{Phases}\label{sec:phases}
This project execution was divided into three distinct phases:
\begin{enumerate}
 \item {\textbf Phase 1}: Team created as of four.
 \item {\textbf Phase 2}: Adopted Agile (Scrum) project management method with three-member team.
 \item {\textbf Phase 3}: Replace Agile with Project Management Lite~\cite{pmlite}.
\end{enumerate}

\subsection{Phase 1}\label{subsec:phase-1}
This project began around October 2021 with a team of four students.
During this phase, the team identified \emph{draw.io} as a good starting point for the project so the team would not have to design a GUI from scratch.

\subsection{Phase 2}\label{subsec:phase-2}
After beginning the project, Frances Joy Poblete disenrolled from HAW to pursue a career in user experience design.
Following her disenrollment, the team had to reevaluate the scope of the project and ensure it was still something that could be accomplished by three team members.
Ultimately, the team decided this was still possible, and the remaining team members decided to adopt an Agile (Scrum) approach to project management.
To accomplish this the team decided to use the industry standard software packages Jira~\cite{jira} and Confluence~\cite{confluence}, both from Atlassian.

Ultimately, delays stemming from Ms. Poblete's departure caused development to be delayed from Winter Semester 2021/22 to Summer Semester 2022.
However, around the start of the semester, one of the team members had to take a semester off from the program and left Germany during that time.
As a result, development during the semester was slow to accommodate that team member's needs.

\subsection{Phase 3}\label{subsec:phase-3}
Once full-time development of the project resumed in the weeks following the end of exams in summer semester 2022, the team quickly concluded Scrum~\cite{scrum} (Agile) project management was not ideal for this project.
Scrum is a great project management style that is very successful in industry software development.
However, a central part of Scrum includes daily ``stand-up'' meetings to synchronize the team's efforts and discuss the previous days' accomplishments.
Since the team was working asynchronously across two continents, this approach was rather untenable.

To replace Scrum, the team chose \emph{Project Management Lite}, which was the required reading for IE5-Scientific and Project Work~\cite{pmlite}.
As a result, the PMLite paperwork is included in this report, but the original Scrum Jira-based Kanban boards are not.